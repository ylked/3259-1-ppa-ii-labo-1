\documentclass[
  french,
  a4paper,
]{scrartcl}

\usepackage[pages=some]{background}
%\ULCornerWallPaper{1}{arc-template.pdf}
\backgroundsetup{firstpage = true, scale = 1, angle = 0, opacity = 1, contents = {\includegraphics[width = \paperwidth, height = \paperheight] {arc-template.pdf}}}

\usepackage{listings}

\usepackage{xcolor}
\definecolor{codegreen}{rgb}{0,0.6,0}
\definecolor{codegray}{rgb}{0.5,0.5,0.5}
\definecolor{codepurple}{rgb}{0.58,0,0.82}
\definecolor{backcolour}{rgb}{0.98,0.98,0.98}

\lstdefinestyle{mystyle}{
    backgroundcolor=\color{backcolour},   
    commentstyle=\color{codegreen},
    keywordstyle=\color{magenta},
    numberstyle=\tiny\color{codegray},
    stringstyle=\color{codepurple},
    basicstyle=\ttfamily\footnotesize,
    breakatwhitespace=false,         
    breaklines=true,                 
    captionpos=b,                    
    keepspaces=true,                 
    numbers=left,                    
    numbersep=5pt,                  
    showspaces=false,                
    showstringspaces=false,
    showtabs=false,                  
    tabsize=2
}
\lstset{style=mystyle}

\usepackage{amsmath,amssymb}
\usepackage[french]{babel}

\usepackage[T1]{fontenc}
\usepackage[utf8]{inputenc}

\usepackage{lmodern}

\usepackage{xcolor}
\usepackage[margin=2cm,includehead,includefoot]{geometry}
\usepackage{longtable,booktabs,array}
\usepackage{calc}
\usepackage[hidelinks]{hyperref}
\usepackage{etoolbox}

\renewcommand{\arraystretch}{1.2}
\setlength {\parindent}{0em}
\setlength {\parskip}{1em}

\usepackage{fancyhdr}
\pagestyle{fancy}

\title{Laboratoire 1 : Comparaison de performances entre une recherche linéaire et une recherche concurrente}
\subject{3259.1 Paradigmes de programmation avancés II $\cdot$ ISC3il-a}
\author{Nima Dekhli\\
    \small \href{mailto:nima.dekhli@he-arc.ch}{nima.dekhli@he-arc.ch}}
\date{\today}


\makeatletter
\providecommand{\subtitle}[1]{% add subtitle to \maketitle
  \apptocmd{\@title}{\par {\large #1 \par}}{}{}
}
\makeatother


\begin{document}
\maketitle
\tableofcontents

\section{Introduction}

Dans le cadre du cours de paradigmes de programmation avancés II, il nous 
a été demandé de réaliser un laboratoire qui consiste à comparer les performances
de deux algorithmes de recherche, à savoir la recherche linéaire et la recherche
concurrente. La recherche concurrente est une recherche dichotomique qui est
exécutée en parallèle sur plusieurs threads, en utilisant un ForkJoinPool.

La recherche s'effectue sur une base de données de films qui nous a été fournie. 
Les méthodes de chargement de la base de données, ainsi que le filtrage 
pour la recherche ont été fournies. Il nous a été demandé de spécifiquement réaliser
la partie de recherche. 

Ce rapport présente la comparaison de performances entre ces deux algorithmes, 
ainsi que l'impact sur la performance du nombre de découpages 
lors de la recherche concurrente.

\section{Implémentation}

\subsection{Recherche linéaire}

La recherche linéaire est effectuée dans la classe 
\lstinline|LinSearch|. Son implémentation est assez simple : 
elle parcourt la base de données, sous forme de tableau, du premier élément 
au dernier et applique la méthode de filtrage sur chacun d'entre eux. 
Cette méthode de filtrage est fournie par la méthode \lstinline|Filtre.filtre()|.
Elle prend en paramètre un élément de la base de données ainsi que la liste des filtres
à appliquer et retourne un booléen indiquant si l'élément correspond aux filtres. 

Dans le cas de la recherche d'un seul élément (méthode \lstinline|LinSearch.trouve()|), après 
avoir trouvé le premier élément correspondant aux filtres, la recherche s'arrête 
immédiatement. 

Dans le cas de la recherche de tous les éléments (méthode \lstinline|LinSearch.trouveTous()|),
à chaque fois qu'un élément correspondant aux filtres est trouvé, il est ajouté à
une liste qui est retournée à la fin de la recherche, quand tous les éléments ont été
parcourus.

\subsection{Recherche concurrente}

La recherche concurrente est effectuée dans la classe \lstinline|ConcSearch|.
Elle utilise un ForkJoinPool pour exécuter la recherche en parallèle sur plusieurs threads.

Chaque tâche récursive, implémentée par la classe \lstinline|FindTask|, étend la 
classe abstraite \lstinline|RecursiveAction|. La méthode \lstinline|compute()| est 
implémentée pour effectuer la recherche. 

La recherche est effectuée de manière dichotomique : tant que l'intervalle de recherche 
de la tâche courante est plus grand que le \textit{batch size}, alors la tâche 
est divisée en deux sous-tâches qui sont exécutées en parallèle, grâce 
à la méthode \lstinline{FindTask.fork()}. La fin de l'exécution des sous-tâches 
est attendue grâce à la méthode \lstinline{FindTask.join()}.

Une fois que les deux sous-tâches sont terminées, les résultats sont récupérés 
et la variable \lstinline|result| est mise à jour. 

Dans le cas de la recherche d'un seul élément (méthode \lstinline|ConcSearch.trouve()|),
il est nécessaire d'arrêter la recherche lorsque un élément correspondant aux filtres
est trouvé. La méthode s'exécutant en parallèle, il s'avère plus complexe que 
dans le cas de la recherche linéaire. 



\section{Résultats}

\subsection{Protocole de tests}

\subsection{Mesures}

\section{Conclusion}
\end{document}